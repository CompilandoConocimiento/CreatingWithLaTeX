
% ****************************************************************************************
% **************      COMPILATION OF ALL MATH FORMULAS THAT I KNOW     *******************
% ****************************************************************************************

% ======================================================
% ===============    COMMANDS    =======================
% ======================================================

    % =====  GENERAL MATH  ==========
    \DeclareMathOperator \Space {\quad}                             %Use: \Space for a cool mega space
    \DeclareMathOperator \MiniSpace {\;}                            %Use: \Space for a cool mini space
    \newcommand \Such {\MiniSpace|\MiniSpace}                       %Use: \Such like in sets

    % =====  NUMBER THEORY  ==========
    \DeclareMathOperator \Naturals {\mathbb{N}}                     %Use: \Naturals por Notation
    \DeclareMathOperator \Integers {\mathbb{Z}}                     %Use: \Integers por Notation
    \DeclareMathOperator \Racionals{\mathbb{Q}}                     %Use: \Racionals por Notation
    \DeclareMathOperator \Reals {\mathbb{R}}                        %Use: \Reals por Notation
    \DeclareMathOperator \Complexs {\mathbb{C}}                     %Use: \Complex por Notation

    % === LINEAL ALGEBRA & VECTORS ===
    \DeclareMathOperator \LinealTransformation {\mathcal{T}}        %Use: \LinealTransformation for a cool T

    \newcommand{\pVector}[1]{                                       %Use: \pVector {Matrix Notation} use parentesis
        \ensuremath{\begin{pmatrix}#1\end{pmatrix}}                 %Example: \pVector{a\\b\\c} or \pVector{a&b&c} 
    }
    \newcommand{\lVector}[1]{                                       %Use: \lVector {Matrix Notation} use a abs 
        \ensuremath{\begin{vmatrix}#1\end{vmatrix}}                 %Example: \lVector{a\\b\\c} or \lVector{a&b&c} 
    }
    \newcommand{\Vector}[1]{                                        %Use: \Vector {Matrix Notation} no parentesis
        \ensuremath{\begin{matrix}#1\end{matrix}}                   %Example: \Vector{a\\b\\c} or \Vector{a&b&c}
    }



% ======================================================
% ==================    MATRIX  ========================
% ======================================================
% You can change the style:
%  bmatrix - Use brackets
%  pmatrix - Use parentesis
%  matrix - Use nothing xD

% And "%" is a new cell
% And "\\" is used to enter a new line
\begin{equation*}
   \begin{bmatrix} 1 &4&3 \\ 0&1&7 \\ 0&0&8 \end{bmatrix}
\end{equation*}



% ======================================================
% ==================    TABLES  ========================
% ======================================================

% Style of a Table
% Remember that "|" means make a vertical line
%  - "c" Center things of this column
%  - "r" Rigth align things of this column

\begin{tabular}{r ||c |c | c |c }
   % And this is the name of all the headers
   &  +,- & . & 0,1..9 & \\ [0.5ex] 
   \hline\hline
  
   %Body
   $ \{ q_0\}$  &  $\{ q_1 \}$  &  $\{ q_2 \}$  & $\{ q_1, q_4\}$ &     \\
   $ \{ q_1 \}$ &  $\emptyset$  &  $\{ q_2 \}$  & $\{ q_1, q_4\}$ &     \\
   $ \{ q_2 \}$ &  $\emptyset$  &  $\emptyset$  & $\{ q_3, q_5\}$ &     \\
   $ \{ q_1 \}$ &  $\emptyset$  &  $\{ q_ 2\}$  & $\{ q_1, q_4\}$ &     \\
   $ \{ q_3 \}$ &  $\emptyset$  &  $\emptyset$  & $\{ q_3, q_5\}$ &     \\
   $ \{ q_2 \}$ &  $\emptyset$  &  $\emptyset$  & $\{ q_3, q_5\}$ &     \\
 
\end{tabular}




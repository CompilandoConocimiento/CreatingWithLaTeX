% ****************************************************************************************
% ************************      NAME OF DOCUMENT      ************************************
% ****************************************************************************************


% =======================================================
% =======   ALL COMMANDS AND RULES FOR DOC   ============
% =======================================================
\documentclass[12pt]{article}                               %Type of docuemtn and size of font
\usepackage[margin=1.0in]{geometry}                         %Margins

\usepackage[spanish]{babel}                                 %Please use spanish
\usepackage[utf8]{inputenc}                                 %Please use spanish 
\usepackage[T1]{fontenc}                                    %Please use spanish

\usepackage{amsthm, amssymb, amsfonts}                      %Make math beautiful
\usepackage[fleqn]{amsmath}                                 %Please make equations left
\decimalpoint                                               %Make math beautiful
\setlength{\parindent}{0pt}                                 %Eliminate ugly indentation

\usepackage{graphicx}                                       %Allow to create graphics
\usepackage{wrapfig}                                        %Allow to create images
\graphicspath{ {Graphics/} }                                %Where are the images :D
\usepackage{listings}                                       %We will be using code here
\usepackage[inline]{enumitem}                               %We will need to enumarate

\usepackage{fancyhdr}                                       %Lets make awesome headers/footers
\renewcommand{\footrulewidth}{0.5pt}                        %We will need this!
\setlength{\headheight}{16pt}                               %We will need this!
\setlength{\parskip}{0.5em}                                 %We will need this!
\pagestyle{fancy}                                           %Lets make awesome headers/footers
\lhead{\footnotesize{\leftmark}}                            %Headers!
\rhead{\footnotesize{\rightmark}}                           %Headers!
\lfoot{Compilando Conocimiento}                             %Footers!
\rfoot{Oscar Rosas}                                         %Footers!

\author{Oscar Andrés Rosas}                                 %Who I am




% =====================================================
% ============        COVER PAGE       ================
% =====================================================
\begin{document}
\begin{titlepage}

    \center
    % ============ UNIVERSITY NAME AND DATA =========
    \textbf{\textsc{\Large Proyecto Compilando Conocimiento}}\\[1.0cm] 
    \textsc{\Large Nombre Materia}\\[1.0cm] 

    % ============ NAME OF THE DOCUMENT  ============
    \rule{\linewidth}{0.5mm} \\[1.0cm]
        { \huge \bfseries Nombre del Documento}\\[1.0cm] 
    \rule{\linewidth}{0.5mm} \\[2.0cm]
    
    % ====== SEMI TITLE ==========
    {\LARGE Nombre Aburrido}\\[7cm] 
    
    % ============  MY INFORMATION  =================
    \begin{center} \large
    \textbf{\textsc{Autor:}}\\
    Rosas Hernandez Oscar Andres
    \end{center}

    \vfill

\end{titlepage}




% ======================================================================================
% ==================================     DOCUMENT  =====================================
% ======================================================================================


% =====================================================
% ============    Criterio de Divergencia    ==========
% =====================================================
\section{Nombre del Articulo}

\subsection{Tema en Especial}
Supón que $a _n > 0$ y que también $b_n > 0$. Osea que ambos terminos siempre seran positivos.

Entonces si:
$\lim_{n \to \infty} \left( \frac{a_n}{b_n} \right) = L$

(Donde obviamente L debe ser positivo y finito)

Si todo esto se cumple entonces alguna de las dos proposiciones deben ser verdad:
\begin{itemize}
    \item Ambas $\Sigma a_n$ y $\Sigma b_n$ divergen.
    \item Ambas $\Sigma a_n$ y $\Sigma b_n$ convergen.
\end{itemize}

% =================
% ==== EJEMPLO ====
% =================
\subsubsection{SubTemas}
Busquemos si la siguiente serie diverge o converge:

\begin{equation}
    \sum_{n = 1}^{\infty} \frac{3n^2 + 2}{(n^2 - 5 )^2} 
\end{equation}
Antes que hacer nada, lo mejor es expandir:
\begin{equation}
    \sum_{n = 1}^{\infty} \frac{3n^2 + 2}{n^4 - 10n^2 + 25} 
\end{equation}

Antes que seguir a nada, vemos si con la prueba de la divergencia podemos mostrar que diverge (para ahorrar trabajo)
 

\cite{Sitio1}


% =====================================================
% ============        BIBLIOGRAPHY   ==================
% =====================================================
\clearpage
\bibliographystyle{plain}
    \begin{thebibliography}{9}

    % ============ REFERENCE #1 ========
    \bibitem{Sitio1} 
        PreguntasMathStack
        \\\texttt{Referencia.com/Sectio}


     

\end{thebibliography}



\end{document}